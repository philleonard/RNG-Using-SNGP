\documentclass[a4paper,10.5pt]{article}
\usepackage[utf8x]{inputenc}
\usepackage[left=2.2cm,top=1.8cm,right=2.2cm,bottom=2cm,nohead,nofoot]{geometry}
\setlength{\footskip}{25pt}

\title{Presentation notes}
\date{}
\begin{document}
\maketitle

\noindent \textbf{Slides 1 - 2}

Introduce the presentation\\

\noindent \textbf{Slide 3}

Talk about the three stages;

\begin{itemize}
\item RNG using GP - Koza's paper ``Evolving a Computer Program to Generate Random Numbers Using the Genetic Programming Paradigm."
\item RNG using SNGP - Jacksons paper ``A New, Node-Focused Model for Genetic Programming."
\item Evaluation. GP vs SNGP and GP + SNGP vs other RNG.
\end{itemize}

\noindent \textbf{Slide 4: DESIGN}

GP overview, workings of single RNG. Walk through flowchart.\\

\noindent \textbf{Slide 5}

Inputs are there for testing the program under different conditions.\\

\noindent \textbf{Slide 6}

All functions have an arity of 2.
All Terminals have an arity of 1.\\

\noindent \textbf{Slide 7}

Fitness of a RNG is the equality in occourance of size 1 to 7 sub sequences in the binary sequence output.\\

\noindent \textbf{Slide 8}

FPR encourages breeding of fittest candidates.

Crossover point chosen between external and internal nodes with probabilities definied by the user. \\

\noindent \textbf{Slide 9}

Outputs written to file on each generation to allow user to manipulate the data. 

Program terminates on max \# of generations or a target fitness value defined by the user.\\

\noindent \textbf{Slide 10}

Same idea of evolving a RNG but using SNGP.

Walk through flow chart\\

\noindent \textbf{Slide 11}

Entire population is a single graph structure.\\

\noindent \textbf{Slide 12}

Makes use of Dynamic Programming, tree outputs stored to be used to form outputs further up.

Each node is the root of a tree\\

\noindent \textbf{Slide 13}

Terminals are added once, functions added to fill ret of population. Successors assigned randomly. Outputs and fitnesses are calculated as nodes are added.\\

\noindent \textbf{Slide 14}

$smut$ - an intenal node is selected randomly and one of it's successors is randomly changed (still deeper).\\
In example the last one $\{+ - * J 0 1 \% 2 3\}$ changes to $\{+ - * J 0 1 2\}$\\
\newpage
\noindent \textbf{Slide 15}
 
Outputs same as GP. Terminates on Max \# gen or when $\sum r_i$ reaches target.\\

\noindent \textbf{Slides 16 - 17 : EVALUATION}

Input parameters which are the same for GP and SNGP are used in test cases. Two values for each param, all combinations = 8 test cases.\\

\noindent \textbf{Slide 18}

Methods for using the output data to compare approaches.\\

\noindent \textbf{Slide 19}

Testing against other PRNG \& TRNG.\\

\noindent \textbf{Slide 20}

Updated project plan. Project is going to plan.
\end{document}